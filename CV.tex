%%%%%%%%%%%%%%%%%%%%%%%%%%%%%%%%%%%%%%%%%
% Medium Length Professional CV
% LaTeX Template
% Version 2.0 (8/5/13)
%
% This template has been downloaded from:
% http://www.LaTeXTemplates.com
%
% Original author:
% Trey Hunner (http://www.treyhunner.com/)
%
% Important note:
% This template requires the resume.cls file to be in the same directory as the
% .tex file. The resume.cls file provides the resume style used for structuring the
% document.
%
%%%%%%%%%%%%%%%%%%%%%%%%%%%%%%%%%%%%%%%%%

%----------------------------------------------------------------------------------------
%	PACKAGES AND OTHER DOCUMENT CONFIGURATIONS
%----------------------------------------------------------------------------------------

\documentclass{resume} % Use the custom resume.cls style

\usepackage[left=0.75in,top=0.6in,right=0.75in,bottom=0.6in]{geometry} % Document margins
\usepackage{fancyhdr}
\usepackage{lastpage}
\usepackage{verbatim}
\renewcommand{\headrulewidth}{0pt}
%\thispagestyle{fancy} 
%\rfoot{\thepage} 
\pagestyle{fancy}  
% \cfoot{\thepage\ of \pageref{LastPage}} 

\name{Ruochen LIN} % Your name


%\address{123 Pleasant Lane \\ City, State 12345}  Your secondary addess (optional)
\address{(+1) 608-628-6375 \\ rlin23@wisc.edu} % Your phone number and email
% \address{1101 University Ave \\ Madison, WI 53706}
% \address {Room 3103, 45 Tuodong Rd \\ Kunming,Yunnan, China, 650011}


\pagenumbering{Roman}
\begin{document}



%----------------------------------------------------------------------------------------
%	Basic information SECTION
%----------------------------------------------------------------------------------------
% \begin {rSection}{Basic Information}
% 
% \begin{tabular}{ @{} >{\bfseries}l @{\hspace{10ex}} l }
% Legal Name & Lin, Ruochen\\
% Academically Known as & Lin, Chester Ruochen\\
% Nationality & Chinese\\
% Birth Date & 27$^{\rm{th}}$ September, 1994\\
% Birth Place & Kunming, Yunnan, China\\
% Residence & Madison, WI, USA
% 
% \end{tabular}
% 
% \end{rSection}

%----------------------------------------------------------------------------------------
%	EDUCATION SECTION
%----------------------------------------------------------------------------------------
\begin{rSection}{Education}



\begin {rSubsection}{University of Wisconsin--Madison}{Madison, WI}{M.S.\em\ in Computer Science}{January 2018 -- December 2019 \em(Expected)}
\item GPA: 3.81
\item Focus: Machine Learning and Optimization
\end {rSubsection}

% \begin {rSubsection}{University of Wisconsin--Madison}{Madison, WI}{M.A.\em\ in Mathematics}{January 2018 -- May 2020 \em(Expected)}
% \item GPA: 4.00
% \item Focus: Numerical Optimization
% \end {rSubsection}

\begin {rSubsection}{University of Wisconsin--Madison}{Madison, WI}{M.S.\em\ in Chemistry}{August 2016 -- December 2017}
\item Advisor: Professor Qiang Cui
\item GPA: 3.75
\item Research Interest: Density Functional Tight-Binding Theory
\end {rSubsection}

\begin {rSubsection}{Nanjing University} {Nanjing, China} {B.S.\em \ in Chemistry} {September 2012 -- June 2016}
% \item Overall GPA: 87.32/100
\item GPA: 88/100
\item Advisors: Professors Shuhua Li and Wei Li

\end {rSubsection}

\end{rSection}

%----------------------------------------------------------------------------------------
% PROJECT SECTION
%----------------------------------------------------------------------------------------
\begin{rSection}{Projects}
\begin {rSubsection}{AR.t: an Interactive Museum Guide for Chazen}{Madison, WI}{\em Software Developer}{August 2017 -- December 2017}
\item Description: We developed an Android app for the Chazen Museum of Arts that let visitors to explore the collections of the museum. We use convolutional neural networks to identify picutures taken by visitors of the artworks and generate introductions with augmented reality visual effects.
\item Image Recognition: Convolutional neural network with Tensorflow (main contributor)
\item Server: AWS-based remote server, written in Python (main contributor)
\item Database: MySQL (contributor)
\item Application: Android Application written in Java (contributor)
\end {rSubsection}
\end{rSection}


%----------------------------------------------------------------------------------------
%	Skill SECTION
%----------------------------------------------------------------------------------------
\begin{rSection}{Professional Skills}

\begin{tabular}{ @{} >{\bfseries}l @{\hspace{6ex}} l }

Programing & C++, Python, Java \\
Scientific Computing & Matlab, Mathematica\\
Machine Learning & TensorFlow \\
% Comput. Chem. Packages & Gaussian, PSI4, Tinker\\
Text Editing& LaTex \\
Language & Chinese, English
\end{tabular}

\end{rSection}

%-- \begin{rSubsection}{The University of Western Australia}{Perth, WA, Australia}{\em UWA Research Training Program}{July -- August 2015}
%-- 
%-- \item Took lessons on academic writing and presentation.
%-- \item Advisor: Assistant Professor Amir Karton.
%-- 
%-- \end {rSubsection}
%-- 
%-- 
%-- \begin {rSubsection}{Michigan State University} {East Lansing, MI, USA} {\em American Language and Culture Summer School} {July  -- August 2014}
%-- 
%-- \item GPA: 4.00/4.00
%-- 
%-- \end {rSubsection}
%-- 
%-- \begin {rSubsection}{Zhejiang University}{Hangzhou, China}{\em Advanced Material Chemistry Summer School} {July 2013}
%-- \item GPA: 4.00/4.00
%-- 
%-- \end {rSubsection}



\begin{rSection}{CS-Related Courses}

\begin{tabular}{ @{} >{\bfseries}l @{\hspace{6ex}} l }

CS 760 & Machine Learning \\
CS 726 & Nonlinear Optimization \\
CS 577 & Introduction to Algorithms \\
CS 513 & Numerical Linear Algebra \\
CS 506 & Software Engineering \\
MSE 760 & Monte Carlo Simulation
\end{tabular}

\end{rSection}

%----------------------------------------------------------------------------------------
% STANDARD TEST SECTION
%----------------------------------------------------------------------------------------

%-- \begin{rSection}{Standard Tests}
%-- 
%-- \begin{tabular}{ @{} >{\bfseries}l @{\hspace{6ex}} l }
%-- TOEFL ibt & Total 115: Reading 30, Listening 30, Speaking 26, Writing 29\\
%-- GRE General & Total 328: Verbal 160 (85\%), Quantitative 168 (95\%), Writing 3.5 (38\%)\\
%-- GRE Chemistry & Scaled Score 870 (92\%)
%-- \end{tabular}
%-- 
%-- \end{rSection}

%----------------------------------------------------------------------------------------
%	Publication SECTION
%----------------------------------------------------------------------------------------
\begin{comment}
\begin{rSection}{Publications}
\textbf{C.R. Lin}, L.-J. Yu, S. Li, A. Karton*, \emph{"
To bridge or not to bridge: The role of sulfuric acid in the Beckmann rearrangement"}, \emph{Chem. Phys. Lett.}, 659, 100-104 (2016). \\
W. Li, Y. Li, \textbf{R. Lin}, S. Li*, \emph{"
	Generalized Energy-Based Fragmentation Approach for Localized Excited States of Large Systems"}, \emph{J. Phys. Chem. A}, 120(48), 9667-9677 (2016).

\end{rSection}
\end{comment}


%----------------------------------------------------------------------------------------
%	LAB EXPERIENCE SECTION
%----------------------------------------------------------------------------------------
\begin{comment}
\begin{rSection}{Research Experience}

\begin{rSubsection}{Nanjing University}{Nanjing, China}{Undergraduate Researcher \emph{(Shuhua Li Group)}}{August 2014 -- Present}

\item Incorporated the popular quantum chemistry package \emph{PSI4} into the \emph{Lower Scaling Quantum Chemistry} package, which is based on the algorithm of \emph{GEBF} and developed by the Shuhua Li group at NJU. I also realized  the TS-searching function in LSQC, utilizing the algorithm in the \emph{Gaussian 09} package.

\end{rSubsection}

\begin{rSubsection}{The University of Western Australia}{Perth, Australia}{Fellowship Research Assistant \emph {(Amir Karton Group)}}{July -- August 2015}
\item Conducted calculations with the computational package of \emph{Gaussian 09} to investigate the mechanism of the \emph{Beckmann Rearrangement}, an organic reaction. It turns out that previous computational investigations all overlooked one transition structure, which is essential in the reaction.
%\item Inspired by the discovery that H$_2$SO$_4$ is an extraordinary catalyst for intramolecular H-transfer with a %cyclic mechanism, we re-examined the reaction mechanisms of the Beckmann Rearrangement (BR) in sulphuric acid. %Previous computational results showed that the rate-determining step of the BR is a 1,2-H shift with direct %H-%transfer. However, it turns out that the H-transfer with the cyclic mechanism not only has lower energy than the %direct one, it has even lower energy than the following rearrangement step. This means that the rate-determining %step is actually the rearrangememt step, contraey to previous ideas. This work established a new reaction mechanism %for the Beckmann Rearrangement in H$_2$SO$_4$.
\end {rSubsection}

\end{rSection}

\end{comment}

%----------------------------------------------------------------------------------------
% ADDITIONAL EXPERIENCE SECTION
%----------------------------------------------------------------------------------------

% \begin {rSection} {Social Experience}
% 
% \begin {rSubsection} {Encyclopedia of China (3rd Edition)} {Nanjing, China} {Compiler \emph{(Chemistry Division)}} {November 2015}
% \item    Wrote and edited theoretical chemistry entries like "Molecular Orbital" and "Molecular Symmetry"
% \end {rSubsection}
% 
% \begin {rSubsection} {RAEST 2015 (Satellite Meeting of the 15th ICQC)} {Nanjing, China} {Volunteer}{June 2015}
% \item  Served as translator and coordinator of the meeting, which turned out to be a great success.
% \end {rSubsection}
% 
% \begin {rSubsection} {NJU Oral History Accociation} {Nanjing, China} {Founding Member \& Interviewer} {July 2013 -- Jan 2014}
% \item Led the project of \emph{Oral History of Southwestern China Serving Regiment Members}
% \item Organised interviews with senior NJU alumni
% \item Editted transcripts of more than 100,000 words
% \end {rSubsection}
% 
% 
% 
% \end {rSection}



%----------------------------------------------------------------------------------------
%	HONOUR SECTION
%----------------------------------------------------------------------------------------

% \begin{rSection}{Selected Awards and Honours}
% 
% Elite Program Scholarships (2013--2015), 25 out of 180 \\
% People Scholarship (2014--2015), top 15\% \\
% NJU Gutian Chemical Scholarship (2013), 5 out of 540 \\
% First Prize in Yunnan Olympiad Competition in Chemistry (2011), 9th out of 2500\\
% Second Prize in Yunnan Olympiad Competition in Physics (2011), 50 out of 3000\\
% 
% 
% \end{rSection}

%----------------------------------------------------------------------------------------

\end{document}
